\documentclass[a4paper]{article}
\usepackage[utf8]{inputenc}
\usepackage[T1]{fontenc}
\usepackage{lmodern}
\usepackage[english,czech]{babel}
\usepackage[margin=2.5cm]{geometry}
\usepackage{tikz}
\usetikzlibrary{calc}
\usepackage{amssymb}	% multline
\usepackage{amsthm}		% theoremstyle
\usepackage[fleqn]{amsmath}	% multline
\usepackage{todonotes}
\usepackage{cancel}
\usepackage[subpreambles=true]{standalone}
\usepackage{adjustbox}

\theoremstyle{definition}
%\newtheorem{exercise}{Exercise}
\newtheorem{priklad}{Příklad}

\renewcommand{\labelenumi}{(\alph{enumi})}
\renewcommand{\labelenumii}{(\roman{enumii})}

\newcommand{\vperp}{\rotatebox[origin=c]{-180}{$\perp$}}

\makeatletter
\newcommand{\specialcell}[1]{\ifmeasuring@#1\else\omit$\displaystyle#1$\ignorespaces\fi}
\makeatother

\newcommand{\convexpath}[2]{
    [   
    create hullcoords/.code={
        \global\edef\namelist{#1}
        \foreach [count=\counter] \nodename in \namelist {
            \global\edef\numberofnodes{\counter}
            \coordinate (hullcoord\counter) at (\nodename);
        }
        \coordinate (hullcoord0) at (hullcoord\numberofnodes);
        \pgfmathtruncatemacro\lastnumber{\numberofnodes+1}
        \coordinate (hullcoord\lastnumber) at (hullcoord1);
    },
    create hullcoords
    ]
    ($(hullcoord1)!#2!-90:(hullcoord0)$)
    \foreach [
    evaluate=\currentnode as \previousnode using \currentnode-1,
    evaluate=\currentnode as \nextnode using \currentnode+1
    ] \currentnode in {1,...,\numberofnodes} {
        let \p1 = ($(hullcoord\currentnode) - (hullcoord\previousnode)$),
        \n1 = {atan2(\y1,\x1) + 90},
        \p2 = ($(hullcoord\nextnode) - (hullcoord\currentnode)$),
        \n2 = {atan2(\y2,\x2) + 90},
        \n{delta} = {Mod(\n2-\n1,360) - 360}
        in 
        {arc [start angle=\n1, delta angle=\n{delta}, radius=#2]}
        -- ($(hullcoord\nextnode)!#2!-90:(hullcoord\currentnode)$) 
    }
}

\begin{document}
\section*{Cvičení 1}
\setcounter{priklad}{0}
\begin{priklad}
    Mějme následující formule.
    \begin{enumerate}
        \item $ (A \leftrightarrow B)\rightarrow (\neg A\wedge C) $
        \item $ (A\rightarrow B)\rightarrow C $
        \item $ A\leftrightarrow B $
        \item $ (A\rightarrow B)\leftrightarrow (A\rightarrow C) $
        \item $ (A\vee B)\wedge (A\vee C) $
        \item $ \left[A\rightarrow (B\vee \neg A)\right]\rightarrow (B\rightarrow A) $
        \item $ \left[(A\vee B)\rightarrow (C\rightarrow A)\right]\leftrightarrow (A\vee B\vee C) $
    \end{enumerate}
    
    Pro každou z nich
    \begin{itemize}
        \item vytvořte pravdivostní tabulku a rozhodněte, zda je formule pravdivá a/nebo splnitelná
        \item převeďte formuli do CNF pomocí pravdivostní tabulky
        \item převeďte formuli do CNF pomocí ekvivalentních úprav
        \item zapište formuli jako množinu klauzulí
    \end{itemize}
\noindent\rule{\linewidth}{.2pt}
    \begin{enumerate}
        \item        
%        \begin{table}[hbt]
%            \centering
            \begin{tabular}{|c|c|c|c|c|c|}
                \hline 
                $ A $ & $ B $ & $ C $ & $ A\leftrightarrow B$ & $\neg A\wedge C $ & $ (A \leftrightarrow B)\rightarrow (\neg A\wedge C) $ \\ 
                \hline 
                1 & 1 & 1 & 0 & 0 & 0 \\
                1 & 1 & 0 & 1 & 0 & 0 \\
                1 & 0 & 1 & 0 & 0 & 1 \\
                1 & 0 & 0 & 0 & 0 & 1 \\
                0 & 1 & 1 & 0 & 1 & 1 \\
                0 & 1 & 0 & 0 & 0 & 1 \\
                0 & 0 & 1 & 1 & 1 & 1 \\
                0 & 0 & 0 & 1 & 0 & 0 \\ 
                \hline 
            \end{tabular} 
%        \caption{Pravdivostní tabulka 1 a)}
%        \end{table}
    
        Pro převod do CNF pomocí tabulky se díváme na řádky s nulou. Výrokové proměnné případně znegujeme, abychom dostali nepravdivou disjunkci proměnných.
        
        CNF z tabulky:
        \[ (\neg A\vee \neg B\vee \neg C)\wedge (\neg A\vee \neg B \vee C)\wedge (A\vee B\vee C) \]
        
        CNF z ekvivalentních úprav: 
        {      
%         \setlength{\mathindent}{0cm}
%         \setlength\abovedisplayskip{-.5em}
        \begin{multline*}
        (A \leftrightarrow B)\rightarrow (\neg A\wedge C) \approx \neg [(\neg A\vee B)\wedge (\neg B\vee A)]\vee (\neg A\wedge C) \approx (A\wedge \neg B)\vee (B\wedge \neg A) \vee (\neg A\wedge C) \approx\\
         [(A\vee B)\wedge \cancel{(A\vee \neg A)}\wedge \cancel{(\neg B\vee B)}\wedge (\neg B\vee \neg A)]\vee (\neg A\wedge C) \approx [(A\vee B)\vee (\neg A\wedge C)]\wedge [(\neg B\vee \neg A)\vee \\
         \vee (\neg A\wedge C)] \approx 
         (A\vee B\vee C) \wedge (\neg B\vee \neg A)\wedge \neg A \wedge (\neg B\vee C) \wedge (\neg A\vee C)
        \end{multline*}
        }
        
        CNF podle WolframAlpha:
        \[ (\neg A\vee \neg B)\wedge (A\vee B\vee C) \]
        
        \item
        
%        \begin{table}[hbt]
%            \centering
            \begin{tabular}{|c|c|c|c|c|}
                \hline 
                 $ A $ & $ B $ & C & $ A\rightarrow B $ & $ (A\rightarrow B)\rightarrow C $ \\
                 \hline 
                 1 & 1 & 1 & 1 & 1 \\ 
                 1 & 1 & 0 & 1 & 0 \\ 
                 1 & 0 & 1 & 0 & 1 \\ 
                 1 & 0 & 0 & 0 & 1 \\ 
                 0 & 1 & 1 & 1 & 1 \\ 
                 0 & 1 & 0 & 1 & 0 \\ 
                 0 & 0 & 1 & 1 & 1 \\ 
                 0 & 0 & 0 & 1 & 0 \\
                 \hline
            \end{tabular}
%        \caption{Pravdivostní tabulka 1 b)}
%        \end{table}
    
        \item
        
%        \begin{table}[hbt]
%            \centering
            \begin{tabular}{|c|c|c|}
                \hline 
                $ A $ & $ B $ & $ A\leftrightarrow B $ \\
                1 & 1 & 1 \\
                1 & 0 & 0 \\
                0 & 1 & 0 \\
                0 & 0 & 1 \\
                \hline
            \end{tabular}
%            \caption{Pravdivostní tabulka 1 c)}
%        \end{table}
    
        \item
        
%        \begin{table}[hbt]
%        \centering
        \begin{tabular}{|c|c|c|c|c|c|}
            \hline 
            $ A $ & $ B $ & $ C $ & $ A\rightarrow B $ & $ A\rightarrow C $ & $ (A\rightarrow B)\leftrightarrow (A\rightarrow C) $ \\
            1 & 1 & 1 & 1 & 1 & 1 \\
            1 & 1 & 0 & 1 & 0 & 0 \\
            1 & 0 & 1 & 0 & 1 & 0 \\
            1 & 0 & 0 & 0 & 0 & 1 \\
            0 & 1 & 1 & 1 & 1 & 1 \\
            0 & 1 & 0 & 1 & 1 & 1 \\
            0 & 0 & 1 & 1 & 1 & 1 \\
            0 & 0 & 0 & 1 & 1 & 1 \\
            \hline
        \end{tabular}
%        \caption{Pravdivostní tabulka 1 d)}
%        \end{table}
    
    \item
    
    \begin{centering}
    \begin{tabular}{|c|c|c|c|c|c|}
        \hline 
        $ A $ & $ B $ & $ C $ & $ A\vee B $ & $ A\vee C $ & $ (A\vee B)\wedge (A\vee C) $ \\
        1 & 1 & 1 & 1 & 1 & 1 \\
        1 & 1 & 0 & 1 & 1 & 1 \\
        1 & 0 & 1 & 1 & 1 & 1 \\
        1 & 0 & 0 & 1 & 1 & 1 \\
        0 & 1 & 1 & 1 & 1 & 1 \\
        0 & 1 & 0 & 1 & 0 & 0 \\
        0 & 0 & 1 & 0 & 1 & 0 \\
        0 & 0 & 0 & 0 & 0 & 0 \\
        \hline
    \end{tabular}
%    \caption{Pravdivostní tabulka 1 e)}
    \end{centering}

    \item

    \begin{tabular}{|c|c|c|c|c|c|}
        \hline 
        $ A $ & $ B $ & $ C $ & $ \left[(A\vee B)\rightarrow (C\rightarrow A)\right] $ & $ A\vee B\vee C $ & $ \left[(A\vee B)\rightarrow (C\rightarrow A)\right]\leftrightarrow (A\vee B\vee C) $ \\
        1 & 1 & 1 & 1 & 1 & 1 \\
        1 & 1 & 0 & 1 & 1 & 1 \\
        1 & 0 & 1 & 1 & 1 & 1 \\
        1 & 0 & 0 & 1 & 1 & 1 \\
        0 & 1 & 1 & 0 & 1 & 0 \\
        0 & 1 & 0 & 1 & 1 & 1 \\
        0 & 0 & 1 & 1 & 1 & 1 \\
        0 & 0 & 0 & 1 & 0 & 0 \\
        \hline
    \end{tabular}
    \end{enumerate}
\end{priklad}

\begin{priklad}
    Které formule implikují následující formule?
    \begin{enumerate}
        \item $ A\wedge B $
        \item $ A\vee B $
        \item $ A\rightarrow B $
        \item $ A\leftrightarrow B $
        \item $ \neg A\wedge \neg B $
        \item $ \neg A $
        \item $ \neg(A\rightarrow B) $
    \end{enumerate}
\noindent\rule{\linewidth}{.2pt}
\begin{enumerate}
    \item c,b,d
    \item $ \emptyset $
    \item $ \emptyset $
    \item c
    \item c,d,f
    \item c
    \item b
\end{enumerate}
\end{priklad}

\begin{priklad}
    Můžeme zakódovat $ n $-bitová čísla $ n $-ticí výrokových proměnných $ A_{n-1},\ldots,A_0 $.
    \begin{enumerate}
        \item Zapište formuli $ \varphi(A_1,A_0,B_1,B_0,C_2,C_1,C_0) $ pro sčítání 2bitových čísel $ \bar{A}+\bar{B} = \bar{C} $.
        \item Zapište formuli $ \varphi(A_{n-1},\ldots,A_0,B_{n-1},\ldots,B_0,C_n,\ldots,C_0) $ pro sčítání $ n $-bitových čísel.
    \end{enumerate}
\end{priklad}

\begin{priklad}
    Pomocí algoritmu DPLL rozhodněte, které z následujících formulí jsou splnitelné.
    \begin{enumerate}
        \item $ \neg\left[(A\rightarrow B)\leftrightarrow (A\rightarrow C)\right] $
        \item $ (A\vee B\vee C)\wedge (B\vee D)\wedge (A\rightarrow D)\wedge (B\rightarrow A) $
        \item $ (A\leftrightarrow B)\rightarrow (\neg A\wedge C) $
        \item $ \left[A\rightarrow (B\vee\neg A)\right]\rightarrow (B\rightarrow A) $
        \item $ \left[(A\vee B)\rightarrow (C\rightarrow A)\right]\leftrightarrow (A\vee B\vee C) $
    \end{enumerate}
\end{priklad}

\begin{priklad}
    Rezoluční metodou rozhodněte, které z následujících formulí jsou pravdivé (tautologie).
    
    \begin{enumerate}
        \item $ (A\wedge \neg B\wedge C)\vee (\neg A\wedge\neg B\wedge\neg C)\vee (B\vee C)\vee (A\wedge \neg C)\vee (\neg A\wedge B\wedge\neg C)\vee (\neg A\wedge\neg B\wedge C) $
        \item $ (A\wedge B)\vee (B\wedge C\wedge D)\vee (\neg A\wedge B)\vee (\neg C\wedge \neg D) $
        \item
        {\setlength{\mathindent}{0cm}
        \setlength\abovedisplayskip{-1.5em}
        \begin{multline*}
        (\neg A\wedge B\wedge \neg C)\vee (\neg B\wedge\neg C\wedge D)\vee (\neg C\wedge \neg D)\vee (A\wedge B)\vee (\neg A\wedge B\wedge C)\vee (\neg A\wedge C)\vee (A\wedge\neg B\wedge C)\vee\\ \vee (A\wedge\neg B\wedge D)
        \end{multline*}
        }
    \end{enumerate}
\end{priklad}

\begin{priklad}
    Mějme konečný automat $ \mathcal{A} $ a vstupní slovo $ w $. Zapište formuli $ \varphi_{\mathcal{A},W} $ splnitelnou právě tehdy, když automat $ \mathcal{A} $ akceptuje slovo $ w $.
\end{priklad}

\begin{priklad}
    Vytvořte hru pro následující množinu Hornových klauzulí a určete výherní oblasti.
    
    \begin{align*}
    B\wedge C\wedge D&\rightarrow A & C\wedge F&\rightarrow B & A\wedge F&\rightarrow C & D&\rightarrow B & A\wedge B&\rightarrow F \\
    E&\rightarrow A & D\wedge E&\rightarrow B & &C & &D &&
    \end{align*}
    
    \begin{figure}[htb]
        \centering
        \begin{tikzpicture}

    \node (A) at (2.5,0) {$ \left\langle A\right\rangle $};
    \node (B) at (-3.5,0) {$ \left\langle B\right\rangle $};
    \node (C) at (0,-1.5) {$ \left\langle C\right\rangle $};
    \node (D) at (0,1) {$ \left\langle D\right\rangle $};
    \node (E) at (0,3) {$ \left\langle E\right\rangle $};
    \node (F) at (2.5,-2.5) {$ \left\langle F\right\rangle $};

    \node (BCDA) at (0,0) {$ [B\wedge C\wedge D\rightarrow A] $};
    \node (1D) at (1.75,1) {$ [1\rightarrow D] $};
    \node (DEB) at (-3,3) {$ [D\wedge E\rightarrow B] $};
    \node (DB) at (-2,1) {$ [D\rightarrow B] $};
    \node (EA) at (3,3) {$ [E\rightarrow A] $};
    \node (1C) at (-2,-0.75) {$ [1\rightarrow C] $};
    \node (AFC) at (5,-1.5) {$ [A\wedge F\rightarrow C] $};    
    \node (ABF) at (0,-2.5) {$ [A\wedge B\rightarrow F] $};
    \node (CFB) at (-1.5,-3.5) {$ [C\wedge F\rightarrow B] $};
    
    \draw[->] (D) to (1D);
    \draw[->] (BCDA) to (D);
    \draw[->] (BCDA) to (B);
    \draw[->] (BCDA) to (C);
    \draw[->] (DB) to (D);
    \draw[->] (DEB) to (D);
    \draw[->] (DEB) to (E);
%    \draw[->] (B) .. controls (-3.5,1) and (-3.5,2) .. (DEB);
    \draw[->] (B) to (DEB);    
    \draw[->] (B) to (DB);
%    \draw[->] (B) .. controls (-3.5,-3) and (-2.5,-4) .. (CFB.west);	% upravit
    \draw[->,bend right] (B) to (CFB);
    \draw[->] (A) to (EA);
    \draw[->] (A) to (BCDA);
    \draw[->] (EA) to (E);
    \draw[->] (C) to (1C);
    \draw[->] (C) to (AFC);
%    \draw[->] (ABF.west) .. controls (-2,-3) and (-3.5,-1) .. ([xshift=2]B);
    \draw[->,bend left] (ABF.west) to (B);
    \draw[->] (ABF) to (A);
    \draw[->] (AFC) to (A);
    \draw[->] (AFC) to (F);
    \draw[->] (CFB) to (F);
    \draw[->] (CFB) to (F);
    \draw[->,bend left] (CFB) to (C.west);
    \draw[->] (F) to (ABF);
    
    \draw[thick,red] \convexpath{EA,DEB}{1cm};
    \draw[thick,green] \convexpath{1D,AFC,CFB,B,DB}{1cm};
\end{tikzpicture}

        \caption{Výherní oblasti k příkladu 7}
    \end{figure}
\end{priklad}

\newpage
\section*{Cvičení 2}
\setcounter{priklad}{0}
\begin{priklad}
    Nechť $ f $ je binární funkční symbol, $ g,h $ jsou unární funkční symboly a $ c $ je konstanta.
    \begin{enumerate}
        \item Najděte nejobecnější unifikátor pro následující dvojice termů.
        \begin{enumerate}
            \item $ f(g(x),y) \mbox{\quad a\quad} f(x,h(y))$
            \item $ f(h(x),y) \mbox{\quad a\quad} f(x,h(y))$
            \item $ f(x,f(x,g(y))) \mbox{\quad a\quad} f(y,f(h(c),x))$
            \item $ f(f(x,c),g(f(y,x)) \mbox{\quad a\quad} f(x,g(y))$
        \end{enumerate}
        \item Vyřešte následující množinu rovnic termů.
        \[ x=f(y,z),\quad y=g(u),\quad z=h(y),\quad u=f(v,w),\quad v=f(c,w) \]
    \end{enumerate}
\noindent\rule{\linewidth}{.2pt}
    \todo{Ověřit správnost 2. příkladu}
    \begin{enumerate}
        \item Najděte nejobecnější unifikátor pro následující dvojice termů.
        \begin{enumerate}
            \item nelze unifikovat
            \item nelze unifikovat
            \item nelze unifikovat
            \item nelze unifikovat
        \end{enumerate}
        \item $ x=f(g(f(f(c,w),w)),h(g(f(f(c,w),w)))) $
    \end{enumerate}
\end{priklad}

\begin{priklad}
    Uvažte následující formule
    \begin{enumerate}
        \item $ \exists x\exists y\forall z [z=x \vee z=y] $
        \item $ \forall x[\exists yR(x,y) \rightarrow \exists yR(y,x)] $
        \item $ \forall x\left[\forall y\exists z[R(x,f(y,z))]\rightarrow \forall y\forall z[R(f(x,y),f(x,z))\vee R(y,z)\right] $
        \item $ \exists x\forall yR(x,y)\wedge\forall x\exists yR(x,y)\wedge \forall x\forall y\left[R(x,y)\rightarrow\exists z\left[R(x,z)\wedge R(z,x)\right]\right] $
    \end{enumerate}

    Každou z nich
    \begin{enumerate}
        \item převeďte do Skolemovy normální formy,
        \item převeďte do množiny klauzulí.
    \end{enumerate}
\noindent\rule{\linewidth}{.2pt}
    \begin{enumerate}
        \item SNF:

        $ \forall z\left[z=a\vee z=b\right] $

        MK:

        $ \left\lbrace z=a, z=b\right\rbrace $

        \item
        {\setlength{\mathindent}{0cm}
        \setlength\abovedisplayskip{-1.5em}
        \begin{multline*}
        \forall x[\exists yR(x,y) \rightarrow \exists yR(y,x)] \approx \forall x[\exists y_1R(x,y_1) \rightarrow \exists y_2R(y_2,x)] \approx
        \forall x[\neg \exists y_1R(x,y_1) \vee \exists y_2R(y_2,x)] \approx\\
        \forall x[\forall y_1\neg R(x,y_1) \vee \exists y_2R(y_2,x)] \approx
        \forall x\forall y_1\exists y_2[\neg R(x,y_1) \vee R(y_2,x)] \approx\\
        \forall x\forall y_1[\neg R(x,y_1) \vee R(f(x,y_1),x)]
        \end{multline*}
        }


        PNF:

        $ \forall x\forall y_1\exists y_2[\neg R(x,y_1) \vee R(y_2,x)] $

        SNF:

        $ \forall x\forall y_1[\neg R(x,y_1) \vee R(f(x,y_1),x)] $

        MK:

        $ \lbrace\neg R(x,y_1), R(f(x,y_1),x)\rbrace $

        \item
       {\setlength{\mathindent}{0cm}
        \setlength\abovedisplayskip{-1.5em}
        \begin{multline*}
        \forall x[\forall y\exists z[R(x,f(y,z))]\rightarrow \forall y\forall z[R(f(x,y),f(x,z))\vee R(y,z)] \approx\\
         \forall x[\neg \forall y_1\exists z_1[R(x,f(y_1,z_1))]\vee\forall y_2\forall z_2[R(f(x,y_2),f(x,z_2))\vee R(y_2,z_2)] \approx\\
         \forall x \exists y_1\forall z_1\forall y_2\forall z_2[\neg R(x,f(y_1,z_1))\vee R(f(x,y_2),f(x,z_2))\vee R(y_2,z_2)] \approx\\
         \forall x\forall z_1\forall y_2\forall z_2[\neg R(x,f(f(x),z_1))\vee R(f(x,y_2),f(x,z_2))\vee R(y_2,z_2)]
        \end{multline*}}

        PNF:

        $ \forall x \exists y_1\forall z_1\forall y_2\forall z_2[\neg R(x,f(y_1,z_1))\vee [R(f(x,y_2),f(x,z_2))\vee R(y_2,z_2)] $

        SNF:

        $ \forall x\forall z_1\forall y_2\forall z_2[\neg R(x,f(f(x),z_1))\vee [R(f(x,y_2),f(x,z_2))\vee R(y_2,z_2)] $

        MK:

        $ \lbrace\neg R(x,f(f(x),z_1)), R(f(x,y_2),f(x,z_2)), R(y_2,z_2)\rbrace $

        \item
        {\setlength{\mathindent}{0cm}
        \setlength\abovedisplayskip{-1.5em}
        \begin{multline*}
        \exists x\forall yR(x,y)\wedge\forall x\exists yR(x,y)\wedge \forall x\forall y\left[R(x,y)\rightarrow\exists z\left[R(x,z)\wedge R(z,x)\right]\right] \approx\\
        \exists x_1\forall y_1R(x_1,y_1)\wedge\forall x_2\exists y_2R(x_2,y_2)\wedge \forall x_3\forall y_3\left[\neg R(x_3,y_3)\vee\exists z\left[R(x_3,z)\wedge R(z,x_3)\right]\right] \approx\\
        \exists x_1\forall y_1\forall x_2\exists y_2\forall x_3\forall y_3\exists z[R(x_1,y_1)\wedge R(x_2,y_2)\wedge \left[\neg R(x_3,y_3)\vee R(x_3,z)\right]\wedge[\neg R(x_3,y_3)\vee R(z,x_3)]]\approx\\
        \forall y_1\forall x_2\forall x_3\forall y_3[R(a,y_1)\wedge R(x_2,f(x_2))\wedge \left[\neg R(x_3,y_3)\vee R(x_3,f(x_3,y_3))\right]\wedge\\ \wedge[\neg R(x_3,y_3)\vee R(f(x_3,y_3),x_3)]]
        \end{multline*}
        }

        PNF:

        $ \exists x_1\forall y_1\forall x_2\exists y_2\forall x_3\forall y_3\exists z\left[R(x_1,y_1)\wedge R(x_2,y_2)\wedge \left[\neg R(x_3,y_3)\vee R(x_3,z)\right]\wedge\left[\neg R(x_3,y_3)\vee R(z,x_3)\right]\right] $

        SNF:

        {\setlength{\mathindent}{0cm}
        \setlength\abovedisplayskip{-1.5em}
        \begin{multline*}
        \forall y_1\forall x_2\forall x_3\forall y_3[R(a,y_1)\wedge R(x_2,f(x_2))\wedge \left[\neg R(x_3,y_3)\vee R(x_3,f(x_3,y_3))\right]\wedge \\\wedge[\neg R(x_3,y_3)\vee R(f(x_3,y_3),x_3)]]
        \end{multline*}
        }

        MK:

        $ \lbrace\lbrace R(a,y_1)\rbrace,\lbrace R(x_2,f(x_2))\rbrace, \lbrace\neg R(x_3,y_3), R(x_3,f(x_3,y_3))\rbrace,\lbrace\neg R(x_3,y_3), R(f(x_3,y_3),x_3)\rbrace\rbrace $
        
    \end{enumerate}
\end{priklad}

\begin{priklad}
    Pomocí rezoluční metody zjistěte, zda jsou následující formule sporné.
%    Use the resolution method to check that the following formulae are inconsistent
    \begin{enumerate}
        \item $ \forall x\forall y[x\leq y\rightarrow (P(x)\leftrightarrow P(y))]\wedge \forall x\forall y[x\leq y\vee y\leq x]\wedge\exists xP(x)\wedge\exists x\neg P(x) $
        \item $ \forall x\exists y[y\leq x \wedge\neg E(x,y)]\wedge\forall x\forall y[x\leq y\wedge y\leq x\rightarrow E(x,y)]\wedge \exists x\forall y[x\leq y] $
        \item $ \forall x\forall y[R(x,y)\rightarrow (P(x)\leftrightarrow \neg P(y))]\wedge\forall x\forall y[R(x,y)\rightarrow \exists z[R(x,z)\wedge R(z,y)]]\wedge \exists x\exists yR(x,y) $
        \item $ \forall xR(x,f(x))\wedge \forall x\forall y\forall z[R(x,y)\wedge R(y,z)\rightarrow R(x,z)]\wedge\forall x\forall y[E(x,y)\rightarrow\neg R(x,y)]\wedge\exists xE(x,f(f(x))) $
    \end{enumerate}
\noindent\rule{\linewidth}{.2pt}
    \begin{enumerate}
        \item SNF:

        {\setlength{\mathindent}{0cm}
        \setlength\abovedisplayskip{-1.5em}
        \begin{multline*}
        \forall x_1\forall y_1\forall x_2\forall y_2[(x_1\nleq y_1\vee\neg P(x_1)\vee P(y_1))\wedge (x_1\nleq y_1\vee \neg P(y_1)\vee P(x_1))\wedge (x_2\leq y_2\vee y_2\leq x_2)\wedge \\\wedge P(a)\wedge \neg P(b)]
        \end{multline*}
        }

        MK:

        $ \{\{x_1\nleq y_1,\neg P(x_1), P(y_1)\},\{x_1\nleq y_1, \neg P(y_1), P(x_1)\},\{x_2\leq y_2, y_2\leq x_2\},\{P(a)\},\{\neg P(b)\}\} $

        \begin{figure}[htbp]
            \centering
            \begin{tikzpicture}[
grow'=up,
note/.style={font=\footnotesize,black!65},
level/.style={sibling distance=2.7cm},
level distance =1.3cm,
parent anchor=north,
child anchor=south]
\node {nelze dál rezolvovat}
    child {node {$ \neg P(b), P(a) $}
        child {node {$ b\leq a $}
            child {node {$ a\nleq b $}
                child {node {$ a\nleq y, P(y_1) $}
                    child {node {$ x_1\nleq y_1, \neg P(x_1), P(y_1) $}}
                    child {node {$ P(a) $}}}
                child {node {$ \neg P(b) $}}}
            child {node {$ x_2\leq y_2, y_2\leq x_2 $}}}
        child {node {$ x_1\nleq y_1,\neg P(y_1),P(x_1) $}}};
\end{tikzpicture}
            \caption{Strom k příkladu 3 a)}
        \end{figure}

        Závěr: Formule je bezesporná (konzistentní).

        \item SNF:

        $ \forall x_1\forall x_2\forall y_2\forall y_3[f(x_1)\leq x_1\wedge\neg E(x_1,f(x_1))\wedge [x_2\nleq y_2\vee y_2\nleq x_2\vee E(x_2,y_2)]\wedge [a\leq y_3]] $

        MK:

        $ \{\{f(x_1)\leq x_1\},\{\neg E(x_1,f(x_1))\},\{x_2\nleq y_2, y_2\nleq x_2, E(x_2,y_2)\},\{a\leq y_3\}\} $

        \begin{figure}[htbp]
            \centering
            \begin{tikzpicture}[
grow'=up,
note/.style={font=\footnotesize,black!65},
level/.style={sibling distance=3.5cm},
level distance =1.3cm,
parent anchor=north,
child anchor=south]
\node {$\Box$}
    child {node {$ x_2\nleq f(x_2) $}
        child {node {$ x_2\nleq f(x_2),f(x_2)\nleq x_2 $}
            child {node {$ \neg E(x_1,f(x_1)) $}}
            child {node {$ x_2\nleq y_2,y_2\nleq x_2,E(x_2,y_2) $}}}
        child {node {$ f(x_1)\leq x_1 $}}}
    child {node {$ a\leq y_3 $}};
\end{tikzpicture}
            \caption{Strom k příkladu 3 b)}
        \end{figure}

        Závěr: Formule je sporná (nekonzistentní).

        \item SNF:
        
        {\setlength{\mathindent}{0cm}
        \setlength\abovedisplayskip{-2.5em}
        \begin{multline*}
        \forall x_1\forall y_1 \forall x_2\forall y_2 [[\neg R(x_1,y_1)\vee \overbrace{\neg P(x_1)\vee\neg P(y_1)}^{\text{můžeme unifikovat}}]\wedge [\neg R(x_1,y_1)\vee \overbrace{P(y_1)\vee P(x_1)}^{\text{můžeme unifikovat}}]\wedge[\neg R(x_2,y_2)\vee\\
        \specialcell{\hfill \vee R(x_2,f(x_2,y_2))]\wedge[\neg R(x_2,y_2)\vee R(f(x_2,y_2),y_2)]\wedge R(a,b)] \approx} \\
        \forall x_1 \forall x_2\forall y_2 [[\neg R(x_1,x_1)\vee \neg P(x_1)]\wedge [\neg R(x_1,x_1)\vee P(x_1)]\wedge[\neg R(x_2,y_2)\vee\\ \vee R(x_2,f(x_2,y_2))]\wedge[\neg R(x_2,y_2)\vee R(f(x_2,y_2),y_2)]\wedge R(a,b)]
        \end{multline*}
        }

        MK:
         
        {\setlength{\mathindent}{0cm}
        \setlength\abovedisplayskip{-1.5em}
        \begin{multline*}
        \{\{\neg R(x_1,x_1), \neg P(x_1)\}, \{\neg R(x_1,x_1), P(x_1)\}, \{\neg R(x_2,y_2), R(x_2,f(x_2,y_2))\},\\ \{\neg R(x_2,y_2), R(f(x_2,y_2),y_2)\}, \{R(a,b)\}\}
        \end{multline*}
        }

        \begin{figure}[htbp]
            \centering
            \begin{tikzpicture}[
grow'=up,
note/.style={font=\footnotesize,black!65},
level/.style={sibling distance=4cm},
level distance =1.3cm,
parent anchor=north,
child anchor=south]
\node {nelze dál rezolvovat}
    child {node {$ \neg R(x_1,x_1),\neg R(x_1,x_1) \equiv \neg R(x_1,x_1) $}
        child {node {$ \neg R(x_1,x_1), \neg P(x_1) $}}
        child {node {$ \neg R(x_1,x_1),P(x_1) $}}};
\end{tikzpicture}
            \caption{Strom k příkladu 3 c)}
        \end{figure}

        Závěr: Formule je bezesporná (konzistentní).

        \item SNF:

        {\setlength{\mathindent}{0cm}
        \setlength\abovedisplayskip{-1.5em}
        \begin{multline*}
        \forall x_1\forall x_2\forall y_2\forall z_2\forall x_3\forall y_3[R(x_1,f(x_1))\wedge [\neg R(x_2,y_2)\vee \neg R(y_2,z_2)\vee R(x_2,z_2)]\wedge \\\wedge[\neg E(x_3,y_3)\vee\neg R(x_3,y_3)]\wedge E(a,f(f(a)))]
        \end{multline*}
        }

        MK:

        $ \{ \{R(x_1,f(x_1))\}, \{\neg R(x_2,y_2), \neg R(y_2,z_2), R(x_2,z_2)\},\{\neg E(x_3,y_3),\neg R(x_3,y_3)\},\{E(a,f(f(a)))\} \} $

        \begin{figure}[htbp]
            \centering
            \begin{tikzpicture}[
grow'=up,
note/.style={font=\footnotesize,black!65},
level/.style={sibling distance=4.5cm},
level distance =1.3cm,
parent anchor=north,
child anchor=south]
\node {$\Box$}
    child {node {$ \neg E(x_1,f(f(x_1))) $}
        child {node {$ \neg R(f(x_1),z_2),\neg E(x_1,z_2) $}
            child {node {$ \neg R(f(x_1),z_2),R(x_1,z_2) $}
                child {node {$ \neg R(x_2,y_2),\neg E(y_2,z_2), R(x_2,z_2) $}}
                child {node {$ R(x_1,f(x_1)) $}}}
            child {node {$ \neg E(x_3,y-3),\neg R(x_3,y_3) $}}}
        child {node {$ R(x_1,f(x_1)) $}}}
    child {node {$ E(a,f(f(a))) $}};
\end{tikzpicture}
            \caption{Strom k příkladu 3 d)}
        \end{figure}

        Závěr: Formule je sporná (nekonzistentní).
    \end{enumerate}
\end{priklad}

\begin{priklad}
    Pomocí SLD rezoluce zjistěte, zda je následující množina Hornových klauzulí sporná.
\end{priklad}
\begin{enumerate}
    \item $ \forall xT(x,x), $\\
    $ \forall x\forall y\forall z\left[E(x,y)\wedge T(y,z)\rightarrow T(x,z)\right], $\\
    $ E(a,b), $\\
    $ E(b,c), $\\
    $ E(c,d), $\\
    $ \neg T(a,d). $
    \item $ \forall xT(x,x), $\\
    $ \forall x\forall y\forall z\left[T(x,y)\wedge E(y,z)\rightarrow T(x,z)\right], $\\
    $ E(a,b), $\\
    $ E(b,c), $\\
    $ E(c,d), $\\
    $ \neg T(a,d). $
    \item $ R(c,c), $\\
    $ \forall x\forall y\forall z\left[R(x,f(y,z))\rightarrow R(f(y,x),z)\right], $\\
    $ \neg\forall x\forall y\left[R(f(x,f(y,c))),f(y,f(x,c))\right]. $
\end{enumerate}
\noindent\rule{\linewidth}{.2pt}
\begin{enumerate}
    \item 1. $ \forall xT(x,x) $\\
    2. $ \forall x\forall y\forall z\left[\neg E(x,y)\vee \neg T(y,z)\vee T(x,z)\right] $\\
    3. $ E(a,b) $\\
    4. $ E(b,c) $\\
    5. $ E(c,d) $\\
    $ \neg T(a,d) $

    \begin{figure}[htbp]
        \centering
        \begin{tikzpicture}[
grow=down,
note/.style={font=\footnotesize,black!65},
level distance=1.3cm]
\node {$\neg T(a,d)$}
    child {node {$ \neg E(a,y),\neg T(y,d) $}
        child {node {$ \neg T(c,d) $}
            child {node {$ \neg E(c,y),\neg T(y,d) $}
                child {node {$ \neg T(d,d) $}
                    child {node {$ \Box $}
                        edge from parent
                            node[note,left] {$ 1) $}}
                    edge from parent
                        node[note,left] {$ 5) $}
                        node[note,right] {$ y/d $}}
                edge from parent
                    node[note,left] {$ 2) $}
                    node[note,right,align=left] {$ x/c, z/d $}}
            edge from parent
                node[note,left] {$ 3) $}
                node[note,right] {$ y/b $}}
        edge from parent
            node[note,left] {$ 2) $}
            node[note,right,align=left] {$ x/a, z/d $}};
\end{tikzpicture}
        \caption{Strom k příkladu 4 a)}
    \end{figure}

    \item 1. $ \forall xT(x,x) $\\
    2. $ \forall x\forall y\forall z\left[\neg T(x,y)\vee \neg E(y,z)\vee T(x,z)\right] $\\
    3. $ E(a,b) $\\
    4. $ E(b,c) $\\
    5. $ E(c,d) $\\
    $ \neg T(a,d) $

    \begin{figure}[htbp]
        \centering
        \begin{tikzpicture}[
grow=down,
note/.style={font=\footnotesize,black!65},
level 2/.style={sibling distance=5cm},
level 3/.style={sibling distance=5cm}]
\node {$\neg T(a,d)$}
    child {node {$ \neg T(a,y),\neg E(y,d) $}
        child {node {$ \neg E(a,d) $}
            child {node {$ fail $}}
            edge from parent
                node[note,left,xshift=-.2cm] {$ 1) $}
                node[note,right] {$ y/a $}}
        child {node {$ \neg T(a,y),\neg E(y,u),\neg E(u,d) $}
            child {node {$ \neg E(a,u),\neg E(u,d) $}
                child {node {$ \neg E(b,d) $}
                    child {node {$ fail $}}                    
                    edge from parent
                        node[note,left] {$ 3) $}
                        node[note,right] {$ u/b $}}
                edge from parent
                    node[note,left,xshift=-.2cm] {$ 1) $}
                    node[note,right,xshift=.1cm] {$ y/a $}}
            child {node {$ \neg T(a,y),\neg E(y,v),\neg E(v,u),\neg E(u,d) $}
                child {node {$ \neg E(a,v),\neg E(v,u),\neg E(u,d) $}
                    child {node {$ \neg E(b,u),\neg E(u,d) $}
                        child {node {$ \neg E(c,d) $}
                            child {node {$ \Box $}
                                edge from parent
                                    node[note,left] {$ 5) $}}
                            edge from parent
                                node[note,left] {$ 4) $}
                                node[note,right] {$ u/c $}}
                        edge from parent
                            node[note,left] {$ 3) $}
                            node[note,right] {$ v/b $}}
                    edge from parent
                        node[note,left] {$ 1) $}
                        node[note,right] {$ y/a $}}
                edge from parent
                    node[note,left,xshift=-.3cm] {$ 2) $}
                    node[note,right,align=right,xshift=.cm] {ren: $ y/v $\\$ x/a $\\$ z/v $}}
            edge from parent
                node[note,left,xshift=-.2cm] {$ 2) $}
                node[note,right,align=right] {ren: $ y/u $\\$ x/a $\\$ z/u $}}
        edge from parent
            node[note,left] {$ 2) $}
            node[note,right,align=left] {$ x/a $\\$ z/d $}
        };
\end{tikzpicture}
        \caption{Strom k příkladu 4 b)}
    \end{figure}

    \item 1. $ R(c,c) $\\
    2. $ \forall x\forall y\forall z\left[\neg R(x,f(y,z))\vee R(f(y,x),z)\right] $\\
    $ \neg R(f(a,f(b,c))),f(b,f(a,c)) $

    \begin{figure}[htbp]
        \centering
        \begin{tikzpicture}[
grow=down,
note/.style={font=\footnotesize,black!65},
level distance=1.3cm]
\node {$\neg R(f(a,f(b,c))),f(b,f(a,c))$}
    child {node {$ \neg R(f(b,c),f(a,f(b,f(a,c)))) $}
        child {node {$ \neg R(c,f(b,f(a,f(a,c)))) $}
            child {node {$ fail $}}
            edge from parent
                node[note,left] {$ 2) $}
                node[note,right] {$ y/b,x/c,z/f(a,f(a,c)) $}}
        edge from parent
            node[note,left] {$ 2) $}
            node[note,right] {$ y/a, x/f(b,c), z/f(b,f(a,c)) $}};
\end{tikzpicture}
        \caption{Strom k příkladu 4 c)}
    \end{figure}
\end{enumerate}

\newpage
\section*{Cvičení 3}
\setcounter{priklad}{0}
% priklad 1
\begin{priklad}
    Předpokládejme, že mámem predikát flight(From, To, Time, Price) obsahující informaci o přímých letech včetně místa odletu, místa příletu, délky letu a ceny za letenku. Napište Program v Prologu, který počítá predikát travel(From, To, Stops, Time, Price) zahrnující všechny možnosti letu z jednoho města do jiného s možností přestupů.
    
\noindent\rule{\linewidth}{.2pt}    
\end{priklad}

% priklad 2
\begin{priklad}
    Napište v Prologu predikát fib(N, X), který počítá Fibonacciho posloupnost. Vyhodnoťte fib(3, X) a fib(N, 5).
    
\noindent\rule{\linewidth}{.2pt}    
\end{priklad}

% priklad 3
\begin{priklad}
    Napište v Prologu definice následujících predikátů.
    
    \begin{table}[htb]
      \begin{center}
        \begin{tabular}{r@{\qquad}l}
           length(List, N) & $ N $ je délka Listu \\
           reverse(X, Y) & $ Y $ je převrácený seznam $ X $ \\
           append(X, Y, Z) & $ Z $ je spojení seznamů $ X $ a $ Y $ \\
           map(X, Y) & mapuje seznam $ X = [X_1,\ldots,X_n] $ na $ Y = [f(X_1),\ldots,f(X_n)] $ \\
           fold\_left(X, Y, Z) & mapuje $ Y = [Y_1,\ldots,Y_n] $ na $ Z = f(\ldots f(f(X,Y_1),Y_2)\ldots,Y_n) $ \\
           fold\_right(X, Y, Z) & mapuje $ Y = [Y_1,\ldots,Y_n] $ na $ Z = f(Y_1,f(Y_2,\ldots,f(Y_n,X)\ldots)) $
        \end{tabular}
      \end{center}
    \end{table}
    
\noindent\rule{\linewidth}{.2pt}    
\end{priklad}

% priklad 4
\begin{priklad}
    Napištenaivní řadící funkci\\
    \quad \texttt{naive\_sort(X, Y) :- permute(X, Y), sorted(Y).}\\
    implementovanou relacemi
    
    \begin{table}[htb]
      \begin{center}
        \begin{tabular}{r@{\qquad}l}
           sorted(X) & kontroluje, že je seznam $ X $ seřazen \\
           insert(X, Y, Z) & seznam $ Z $ jsme získali přidáním $ X $ na libovolnou pozici do seznamu $ Y $ \\
           permute(X, Y) & seznam $ Y $ je permutací seznamu $ X $
        \end{tabular}
      \end{center}
    \end{table}
    
    Implementujte merge sort využitím relace
    
    \begin{table}[htb]
      \begin{center}
        \begin{tabular}{r@{\qquad}l}
           merge(X, Y, Z) & spojí dva seřazene seznamy $ X $ a $ Y $ do seznamu $ Z $
        \end{tabular}
      \end{center}
    \end{table}
    
\noindent\rule{\linewidth}{.2pt}    
\end{priklad}

% priklad 5
\begin{priklad}
    Uvažujme orientovaný graf $ G = (V,E) $. Vyjádřete relace v relační algebře.
    
    \begin{enumerate}
      \item $ x $ a $ y $ nejsou spojeny hranou
      \item hrana $ (x,y) $ je součástí trojúhelníka
      \item $ x $ má aspoň dva sousedy
      \item každý soused $ x $ je zároveň sousedem $ y $
    \end{enumerate}
    
\noindent\rule{\linewidth}{.2pt}    
\end{priklad}

% priklad 6
\begin{priklad}
    Vyhodnoťte následující program v Datalogu na stromě $ (V,E,P) $ doprava.
    
    \begin{minipage}{0.48\textwidth}
      \begin{align*}
      U &\leftarrow S(x,y)\wedge W(x)\wedge W(y) \\
      W(x) &\leftarrow P(x) \\
      W(x) &\leftarrow E(x,y) \wedge W(y) \\
      S(x,y) &\leftarrow E(z,x)\wedge E(z,y)\wedge x\neq y \\
      R(x,y) &\leftarrow P(x)\wedge x=y \\
      R(x,y) &\leftarrow E(x,z)\wedge R(z,y) \\
      R(x,y) &\leftarrow R(x,z)\wedge E(z,y)
      \end{align*}
    \end{minipage}
    \hfill
    \begin{minipage}{0.3\textwidth}
    \begin{tikzpicture}[
    ->,
    level distance=1cm,
    label distance=-.1cm]
    \node {1}
      child {node {2}
        child {node {4}}
        child {node[label={below:P}] {5}}}
      child {node[label={above right:P}] {3}
        child[missing]
        child{node {6}
          child {node[label={below:P}] {7}}
          child {node {8}}}}
      ;
    \end{tikzpicture}      
    \end{minipage}
    
\noindent\rule{\linewidth}{.2pt}    
\end{priklad}

\newpage
\section*{Cvičení 4}
\setcounter{priklad}{0}
% priklad 1
\begin{priklad}
    Použitím přirozené dedukce dokažte, že následující formule jsou pravdivé.
    
    \begin{enumerate}
      \item $ \varphi\wedge \psi\rightarrow\psi\wedge\varphi $
      \item $ \psi\rightarrow ((\varphi\wedge\psi)\vee\psi) $
      \item $ (\neg \psi\rightarrow \neg\varphi) \rightarrow (\varphi\rightarrow\psi) $
      \item $ \varphi \rightarrow \neg\neg\varphi $
      \item $ ((\varphi\wedge\psi)\vee\psi) \rightarrow \psi $
      \item $ \neg\neg\varphi \rightarrow \varphi $
      \item $ \varphi\vee \neg\varphi $
      \item $ \neg(\neg\varphi\wedge\neg\psi) \rightarrow (\varphi\vee\psi) $ (Zákeřné)
      \item $ \varphi \rightarrow \exists x\varphi $
      \item $ \forall x\varphi \rightarrow \varphi $
      \item $ \forall xR(x,x) \rightarrow \forall x\exists yR(f(x),y) $
      \item $ \exists x(\varphi\vee\psi) \rightarrow (\exists x\varphi\vee\exists x\psi) $
      \item $ (\exists x\varphi\vee \exists x\psi) \rightarrow \exists x(\varphi\vee\psi) $
      \item $ \forall x\varphi \wedge \forall x\psi \rightarrow \forall x(\varphi\wedge\psi) $
      \item $ \forall x(\varphi\wedge \psi) \rightarrow \forall x\varphi \wedge \forall x\psi $
      \item $ \forall x\forall y[\varphi(x) \leftrightarrow \varphi(y)] \wedge \exists x\varphi(x) \rightarrow \forall x\varphi(x) $
    \end{enumerate}

      \begin{enumerate}
      \item
        \includestandalone[mode=build]{cv-4/cv-4-1a-res}        
      
      \item
        \includestandalone[mode=build]{cv-4/cv-4-1b-res}    
        
      \item
      
      \item
        \includestandalone[mode=build]{cv-4/cv-4-1d-res}
      \item
        \includestandalone[mode=build]{cv-4/cv-4-1e-res}
      \item 
        \includestandalone[mode=build]{cv-4/cv-4-1f-res}
      \item
        \includestandalone[mode=build, width=\linewidth]{cv-4/cv-4-1g-res}
      \item
        \includestandalone[mode=build, width=\linewidth]{cv-4/cv-4-1h-res}
      \item
      \item 
      \item
	    \includestandalone[mode=build]{cv-4/cv-4-1k-res}      
      \item
      \item
      \item 
      \item
      \item
    \end{enumerate}
\end{priklad}

% priklad 2
\begin{priklad}
    Použitím tableau důkazu ukažte, že formule z příkladu 1 jsou pravdivé.

\noindent\rule{\linewidth}{.2pt}

    \begin{enumerate}
      \item
        \includestandalone[mode=buildmissing]{cv-4/cv-4-2a-res}
      \item        
      \item      
        \includestandalone[mode=buildmissing]{cv-4/cv-4-2c-res}
      \item
      \item      
        \includestandalone[mode=buildmissing]{cv-4/cv-4-2e-res}
      \item 

      \item

      \item

      \item
      \item 
      \item
      \item
      \item
      \item 
      \item
      \item
    \end{enumerate}
\end{priklad}

% priklad 3
\begin{priklad}
    Najděte všechny konzistentní množiny pro následující množiny pravidel.
    
    \begin{enumerate}
      \item $\displaystyle \frac{}{\alpha} \quad \frac{\alpha:\beta}{\delta} \quad \frac{\alpha:\gamma}{\delta}$
      \item $\displaystyle \frac{}{\alpha} \quad \frac{\alpha:\beta\gamma}{\beta}$
      \item $\displaystyle \frac{}{\alpha} \quad \frac{\alpha\beta}{\gamma} \quad \frac{\alpha:\gamma}{\beta}$
    \end{enumerate}
    
\noindent\rule{\linewidth}{.2pt}    
\end{priklad}

% priklad 4
\begin{priklad}
    Pro každou podmnožinu $ \Phi\subseteq \mathcal{P}(\{\alpha,\beta\}) $ se pokuste najít množinu pravidel $ R $ takovou, že $ \Phi $ je množinou všech konzistentních množin pro $ R $.
    
\noindent\rule{\linewidth}{.2pt}    
\end{priklad}

% priklad 5
\begin{priklad}
    Ze základních pravidel přirozené dedukce odvoďte následující pravidla.
    
    \[
    \frac{\Gamma\vdash \neg\neg\varphi}{\Gamma\vdash\varphi} \qquad 
    \frac{\Gamma\vdash \varphi}{\Gamma,\Delta\vdash\varphi} \qquad
    \frac{\Gamma,\neg\varphi\vdash \neg\psi}{\Gamma\vdash\psi\rightarrow\varphi} \qquad
    \frac{\Gamma,\neg\varphi\vdash \psi}{\Gamma\vdash\varphi\vee\psi}
    \]
    
\noindent\rule{\linewidth}{.2pt}    
\end{priklad}

% priklad 6
\begin{priklad}
    Najděte pravidlo pro důkazy indukcí.
    
    \[ \frac{\ldots\quad\ldots}{\forall x\varphi(x)} \qquad \text{kde } \varphi(x) \text{ je formule mluvící o přirozených číslech.} \]
    
\noindent\rule{\linewidth}{.2pt}    
\end{priklad}

\newpage
\section*{Cvičení 5}
\setcounter{priklad}{0}
% priklad 1
\begin{priklad}
    Spusťte Find-S algoritmus na následujících vstupech.
    
    \begin{enumerate}
      \item 
          \begin{tabular}{*{8}{c}|c}
          \hline
           $ x_0 $ & $ x_1 $ & $ x_2 $ & $ x_3 $ & $ x_4 $ & $ x_5 $ & $ x_6 $ & $ x_7 $ & $ f(\bar{x}) $ \\
           \hline
           $ 0 $ & $ 1 $ & $ 1 $ & $ 1 $ & $ 0 $ & $ 1 $ & $ 1 $ & $ 0 $ & $ 1 $ \\
           $ 1 $ & $ 1 $ & $ 1 $ & $ 0 $ & $ 0 $ & $ 1 $ & $ 0 $ & $ 0 $ & $ 1 $ \\
           $ 1 $ & $ 1 $ & $ 1 $ & $ 0 $ & $ 1 $ & $ 1 $ & $ 1 $ & $ 1 $ & $ 0 $ \\
           $ 0 $ & $ 1 $ & $ 0 $ & $ 1 $ & $ 0 $ & $ 1 $ & $ 1 $ & $ 0 $ & $ 1 $ \\
           $ 0 $ & $ 1 $ & $ 0 $ & $ 1 $ & $ 0 $ & $ 1 $ & $ 0 $ & $ 0 $ & $ 1 $ \\
           $ 1 $ & $ 0 $ & $ 0 $ & $ 0 $ & $ 1 $ & $ 0 $ & $ 1 $ & $ 1 $ & $ 0 $ \\
           $ 0 $ & $ 0 $ & $ 0 $ & $ 1 $ & $ 0 $ & $ 0 $ & $ 0 $ & $ 1 $ & $ 0 $ \\
           $ 1 $ & $ 1 $ & $ 1 $ & $ 0 $ & $ 0 $ & $ 1 $ & $ 0 $ & $ 0 $ & $ 1 $ \\
           \hline
          \end{tabular}
      \item 
          \begin{tabular}{*{8}{c}|c}
          \hline
           $ x_0 $ & $ x_1 $ & $ x_2 $ & $ x_3 $ & $ x_4 $ & $ x_5 $ & $ x_6 $ & $ x_7 $ & $ f(\bar{x}) $ \\
           \hline
           $ 0 $ & $ 1 $ & $ 0 $ & $ 1 $ & $ 1 $ & $ 0 $ & $ 0 $ & $ 1 $ & $ 1 $ \\
           $ 0 $ & $ 1 $ & $ 1 $ & $ 1 $ & $ 0 $ & $ 0 $ & $ 1 $ & $ 1 $ & $ 0 $ \\
           $ 1 $ & $ 1 $ & $ 0 $ & $ 0 $ & $ 1 $ & $ 1 $ & $ 0 $ & $ 1 $ & $ 0 $ \\
           $ 0 $ & $ 1 $ & $ 1 $ & $ 0 $ & $ 1 $ & $ 0 $ & $ 0 $ & $ 1 $ & $ 1 $ \\
           $ 0 $ & $ 0 $ & $ 0 $ & $ 1 $ & $ 0 $ & $ 0 $ & $ 0 $ & $ 0 $ & $ 1 $ \\
           $ 1 $ & $ 0 $ & $ 0 $ & $ 0 $ & $ 1 $ & $ 0 $ & $ 1 $ & $ 1 $ & $ 0 $ \\
           $ 0 $ & $ 0 $ & $ 1 $ & $ 1 $ & $ 1 $ & $ 0 $ & $ 0 $ & $ 0 $ & $ 1 $ \\
           $ 0 $ & $ 1 $ & $ 1 $ & $ 0 $ & $ 0 $ & $ 0 $ & $ 0 $ & $ 1 $ & $ 1 $ \\
           \hline
          \end{tabular}
    \end{enumerate}
    
\noindent\rule{\linewidth}{.2pt}

\begin{enumerate}
  \item %
  \begin{align*}
  h_0 &= \perp \\
  h_1 &= \neg x_0 \wedge x_1 \wedge x_2 \wedge x_3 \wedge \neg x_4 \wedge x_5 \wedge x_6 \wedge \neg x_7 \\
  h_2 &= x_1 \wedge x_2 \wedge \neg x_4 \wedge x_5 \wedge \neg x_7 \\
  h_3 &= x_1 \wedge \neg x_4 \wedge x_5 \wedge \neg x_7 \\
  h_4 &= x_1 \wedge \neg x_4 \wedge x_5 \wedge \neg x_7 \\
  h_5 &= x_1 \wedge \neg x_4 \wedge x_5 \wedge \neg x_7 \\
\end{align*}
  
  \item %
  \begin{align*}
  h_0 &= \perp \\
  h_1 &= \neg x_0 \wedge x_1 \wedge \neg x_2 \wedge x_3 \wedge x_4 \wedge \neg x_5 \wedge \neg x_6 \wedge x_7 \\
  h_2 &= \neg x_0 \wedge x_1 \wedge x_4 \wedge \neg x_5 \wedge \neg x_6 \wedge x_7 \\
  h_3 &= \neg x_0 \wedge \neg x_5 \wedge \neg x_6 \\
  h_4 &= \neg x_0 \wedge \neg x_5 \wedge \neg x_6 \\
  h_5 &= \neg x_0 \wedge \neg x_5 \wedge \neg x_6 \\
\end{align*}
\end{enumerate}
\end{priklad}

% priklad 2
\begin{priklad}
    Spusťte Candidate-Elimination algoritmus na vstupech.
    
    \begin{enumerate}
      \item 
          \begin{tabular}{*{4}{c}|c}
          \hline
           $ x_0 $ & $ x_1 $ & $ x_2 $ & $ x_3 $ & $ f(\bar{x}) $ \\
           \hline
           $ 0 $ & $ 1 $ & $ 1 $ & $ 0 $ & $ 1 $ \\
           $ 1 $ & $ 1 $ & $ 0 $ & $ 0 $ & $ 1 $ \\
           $ 1 $ & $ 0 $ & $ 1 $ & $ 1 $ & $ 0 $ \\
           $ 0 $ & $ 0 $ & $ 1 $ & $ 0 $ & $ 1 $ \\
           $ 0 $ & $ 0 $ & $ 1 $ & $ 1 $ & $ 0 $ \\
           $ 1 $ & $ 1 $ & $ 0 $ & $ 1 $ & $ 0 $ \\
           \hline
          \end{tabular}
      \item 
          \begin{tabular}{*{4}{c}|c}
          \hline
           $ x_0 $ & $ x_1 $ & $ x_2 $ & $ x_3 $ & $ f(\bar{x}) $ \\
           \hline
           $ 1 $ & $ 0 $ & $ 1 $ & $ 1 $ & $ 0 $ \\
           $ 1 $ & $ 1 $ & $ 0 $ & $ 0 $ & $ 1 $ \\
           $ 0 $ & $ 0 $ & $ 0 $ & $ 0 $ & $ 1 $ \\
           $ 0 $ & $ 1 $ & $ 1 $ & $ 1 $ & $ 0 $ \\
           $ 1 $ & $ 0 $ & $ 0 $ & $ 0 $ & $ 1 $ \\
           $ 0 $ & $ 0 $ & $ 1 $ & $ 1 $ & $ 0 $ \\
           \hline
          \end{tabular}
    \end{enumerate}    
\noindent\rule{\linewidth}{.2pt}  
\begin{enumerate}
  \item 
  \begin{align*}
    0{:} & \text{H}^- = \{ \perp \} & \text{H}^+ &= \{ \vperp \} \\
    1{:} & \text{H}^- = \{ \neg x_0 \wedge x_1 \wedge x_2 \wedge \neg x_3 \} & \text{H}^+ &= \{ \vperp \} \\
    2{:} & \text{H}^- = \{ x_1 \wedge \neg x_3 \} & \text{H}^+ &= \{ \vperp \} \\
    3{:} & \text{H}^- = \{ x_1 \wedge \neg x_3 \} & \text{H}^+ &= \{ x_1, \neg x_3 \} \\    
    4{:} & \text{H}^- = \{ \neg x_3 \} & \text{H}^+ &= \{ \neg x_3 \} \\
    5{:} & \text{H}^- = \{ \neg x_3 \} & \text{H}^+ &= \{ \neg x_3 \} \\
  \end{align*}
\end{enumerate}  
\end{priklad}

\newpage
\section*{Cvičení 6}
\setcounter{priklad}{0}
% priklad 1
\begin{priklad}
    Uvažujme slova nad abecedou $ \{a,b\} $ jako přechodové systémy $ (S,E_s,E_r,P_a,P_b) $, kde stavy $ S $ jsou pozice, dva predikáty $ P_a $ a $ P_b $ označí každou pozici odpovídajících znakem a dvě hranové relace
    \[ E_s = \{(i,i+1) \mid i<n-1\}\mbox{,}\qquad E_r = \{(i,k) \mid i\leq k<n\}\mbox{,} \]
    kde $ n = |S| $ je délka slova. Definujte následující jazyky v modální logice.
    
    \begin{enumerate}
      \item všechna slova začínající na písmeno $ a $
      \item všechna slova se skládají pouze z písmen $ a $
      \item všechna slova končí na písmeno $ a $
      \item $ a^*b^* $
      \item všechna slova obsahují faktor $ bb $
      \item všechna slova obsahují aspoň dvě písmena $ b $
      \item všechna slova obsahují právě dvě písmena $ b $
      \item $ (ab)* $
    \end{enumerate}
    
\noindent\rule{\linewidth}{.2pt}    
\end{priklad}

% priklad 2
\begin{priklad}
    Přepište následující formule do predikátové logiky.
    
    \begin{enumerate}
      \item $ [a]P\rightarrow P $
      \item $ P\rightarrow \langle a\rangle Q $
      \item $ [a](P\wedge \langle b\rangle Q) \rightarrow (\langle a\rangle P\vee\langle b\rangle Q) $
    \end{enumerate}
    
\noindent\rule{\linewidth}{.2pt}    
\end{priklad}

% priklad 3
\begin{priklad}
    Dokažte následující modální formule pomocí tabel.
    
    \begin{enumerate}
      \item $ \Box (P\leftrightarrow (Q\wedge R)) \rightarrow (\Box P\leftrightarrow (\Box Q\wedge \Box R)) $
      \item $ \neg\Box\Box P\leftrightarrow \Diamond\Diamond\neg P $
      \item $ \Box (P\wedge \neg P)\rightarrow \Box Q $
      \item $ \neg \Diamond P\rightarrow \Box(P\rightarrow Q) $
    \end{enumerate}
    
\noindent\rule{\linewidth}{.2pt}    
\end{priklad}

% priklad 4
\begin{priklad}
    Najděte CTL$ ^* $ formule definující následující vlastnosti $ \{a,b\} $-označených stromů.
    
    \begin{enumerate}
      \item je aspoň jedno označení $ b $
      \item každá cesta obsahuje nějaké $ b $
      \item každá cesta obsahuje aspoň dvě $ b $
      \item všechny cesty obsahují nekonečně mnoho $ b $
      \item nějaká cesta obsahuje nekonečně mnoho $ b $
    \end{enumerate}
    
\noindent\rule{\linewidth}{.2pt}    
\end{priklad}

% priklad 5
% \begin{priklad}
%     Modelujeme výtah v 3patrové budově.
%     
%     \begin{enumerate}
%       \item Popište výtah jako přechodový systém.
%       \item Napište specifikaci pro výtah v modální logice a v LTL. Začněte s následujícími dvěma tvrzeními, poté přidejte vlastní.
%         \begin{enumerate}
%           \item Výtah se nikdy nepohne, když jsou dveře otevřené.
%           \item Je-li
%         \end{enumerate}
%     \end{enumerate}
%     
% \noindent\rule{\linewidth}{.2pt}    
% \end{priklad}

\end{document}
